\documentclass[bookmarks=true,bookmarksopen=true,pdfborder={0 0 0},pdfhighlight={/N},linkbordercolor={.5 .5 .5},implicit=false,unicode,xcolor={table}]{beamer}

\usepackage[T1]{fontenc}
\usepackage[utf8]{inputenc}

\usepackage[english,serbian]{babel}

%\usepackage[unicode]{hyperref}

\mode<presentation>
\setbeamercovered{transparent}
\setbeamertemplate{navigation symbols}{}

\usetheme{Frankfurt}
\usecolortheme{beaver}

\usepackage{color}
%\usepackage[table]{xcolor}
\usepackage{url}

\usepackage{graphicx}
\usepackage{xspace}
\usepackage{fancyvrb}
\usepackage{listings}


\graphicspath{ {./Slike/} }
\usepackage{subcaption}
\captionsetup{compatibility=false}

\definecolor{mygreen}{rgb}{0,0.6,0}
\definecolor{mygray}{rgb}{0.5,0.5,0.5}
\definecolor{mymauve}{rgb}{0.58,0,0.82}

\lstset{ 
  backgroundcolor=\color{white},   % choose the background color; you must add \usepackage{color} or \usepackage{xcolor}; should come as last argument
  basicstyle=\scriptsize\ttfamily,        % the size of the fonts that are used for the code
  breakatwhitespace=false,         % sets if automatic breaks should only happen at whitespace
  breaklines=true,                 % sets automatic line breaking
  captionpos=b,                    % sets the caption-position to bottom
  commentstyle=\color{mygreen},    % comment style
  deletekeywords={...},            % if you want to delete keywords from the given language
  escapeinside={\%*}{*)},          % if you want to add LaTeX within your code
  extendedchars=true,              % lets you use non-ASCII characters; for 8-bits encodings only, does not work with UTF-8
  firstnumber=1,                % start line enumeration with line 1000
  frame=single,	                   % adds a frame around the code
  keepspaces=true,                 % keeps spaces in text, useful for keeping indentation of code (possibly needs columns=flexible)
  keywordstyle=\color{blue},       % keyword style
  language=C,                      % the language of the code
  morekeywords={*,...},            % if you want to add more keywords to the set
  numbers=left,                    % where to put the line-numbers; possible values are (none, left, right)
  numbersep=5pt,                   % how far the line-numbers are from the code
  numberstyle=\tiny\color{mygray}, % the style that is used for the line-numbers
  rulecolor=\color{black},         % if not set, the frame-color may be changed on line-breaks within not-black text (e.g. comments (green here))
  showspaces=false,                % show spaces everywhere adding particular underscores; it overrides 'showstringspaces'
  showstringspaces=false,          % underline spaces within strings only
  showtabs=false,                  % show tabs within strings adding particular underscores
  stepnumber=1,                    % the step between two line-numbers. If it's 1, each line will be numbered
  stringstyle=\color{mymauve},       % string literal style
  tabsize=2,	                   % sets default tabsize to 2 spaces
  title=\lstname                   % show the filename of files included with \lstinputlisting; also try caption instead of title
}

\setbeamertemplate{itemize item}[circle]
\setbeamertemplate{itemize subitem}[square]

\AtBeginSection[]{
  \begin{frame}
  \vfill
  \centering
  \begin{beamercolorbox}[sep=8pt,center,shadow=true,rounded=true]{title}
    \usebeamerfont{title}\insertsectionhead\par%
  \end{beamercolorbox}
  \vfill
  \end{frame}
}


\begin{document}
\title{Model kamere}
\author{Momir Adžemović\\ \small{momir.adzemovic@gmail.com}}
\subtitle{Računarski vid}
\institute{Istraživačka stanica Petnica}
\date{}

\begin{frame}

  \titlepage{}

\end{frame}

\begin{frame}{O meni}

  \begin{itemize}
    \item Saradnik u nastavi i master student na Matematičkom fakultetu u Beogradu
    \item Istraživač u oblasti mašinskog učenja u Everseen-u u Beogradu
  \end{itemize}

  \begin{figure}
		\begin{subfigure}{3cm}
			\includegraphics[width=3cm,height=3cm]{images/matf.png}
		\end{subfigure}
    \hfill
		\begin{subfigure}{3cm}
			\includegraphics[width=3cm,height=3cm]{images/everseen.jpeg}
		\end{subfigure}
	\end{figure}
  \hfill

\end{frame}

\begin{frame}{Računarski vid}

Računarski vid (\textit{eng. Computer Vision}) je ,,automatizovani`` proces izvlačenja informacija iz slika.
\begin{itemize}
  \item Pozicija i orijentacija kamere
  \item Uklananje distorzije sa slika
  \item Detekcija objekata
  \item Prepoznavanje objekata
  \item Estimacija dubine/udaljenosti na slikama
\end{itemize}

\begin{figure}
		
  \begin{subfigure}{4cm}
    \includegraphics[width=4cm,height=3cm]{images/cv_object_detection.png}
  \end{subfigure}
  \begin{subfigure}{4cm}
    \includegraphics[width=4cm,height=3cm]{images/cv_depth_estimation.jpeg}
  \end{subfigure}
  \begin{subfigure}{2cm}
    \includegraphics[width=2cm,height=4cm]{images/cv_instance_segmentation.png}
  \end{subfigure}
\end{figure}

\end{frame}

\begin{frame}{Kako funkcioniše kamera? }
  
  Neophodno je da poznajemo neke osnovne pojmove iz linearne algebre i analitičke geometrije.
  \begin{itemize}
    \item Vektori i matrice
    \item Koordinatni sistem
    \item Afine i projektivne transformacije
  \end{itemize}

  \begin{figure}
    \begin{subfigure}{3cm}
      \includegraphics[width=3cm,height=3cm]{images/camera_classic.png}
    \end{subfigure}
    \begin{subfigure}{3cm}
      \includegraphics[width=3cm,height=3cm]{images/camera_bullet.jpg}
    \end{subfigure}
  \end{figure}

\end{frame}

\section{Osnove analitičke geometrije}
\begin{frame}{Vektori}
  
  \begin{itemize}
    \item \textbf{Vektor} je klasa ekvivalencije usmerenih duži. Svaka klasa ekvivalencije je određena pravcem, smerom i intenzitetom
    \item U ovom slučaju je dovoljno da vektor posmatramo kao niz skalara (vrednosti), a sama formalna definicija nam nije neophodna
    \item Primeri:
      \begin{itemize}
        \item (2, 3)
        \item (1.3, 10.3)
        \item (13, 10, 2)
        \item (1, 1, 1, 1, ..., 1)
        \item (-2, 3)
      \end{itemize}
  \end{itemize}

\end{frame}

\begin{frame}{Vektori - Osnovne operacije i pojmovi}
  
  \begin{itemize}
    \item Sabiranje: $(2, 3) + (1, 1) = (3, 4)$
    \item Oduzimanje: $(2, 3) - (1, 1) = (1, 2)$
    \item Množenje skalarom: $5 * (2, 3) = (10, 15)$
    \item Nula vektor: $(0, 0)$ (neutral za sabiranje i oduzimanje)
    \item Skalarni proizvod: $(2, 3) \cdot (4, 2) = 2*4 + 3*2 = 14$
  \end{itemize}

\end{frame}

\begin{frame}{Tačke}

  \begin{itemize}
    \item Tačke (koordinate) u ravni (2D) i prostoru (3D) možemo da predstavljamo vektorima
    \item Da li nam je jedan vektor dovoljan da opišemo tačku?
    \begin{itemize}
      \item Sama vrednost (2, 3) nam ne znači puno ako ne znamo u odnosu na šta gledamo
      \item Neophodno je da definišemo centar našeg sistema koordinata
      \item Tada svaki vektor predstavlja rastojanje od centra koordinatnog sistema po svakoj od osa
      \item \textbf{Kartezijev koordinatni sistem}
    \end{itemize}
  \end{itemize}

  \begin{figure}
    \begin{subfigure}{4cm}
      \includegraphics[width=4cm,height=4cm]{images/cartesian_coordinate_system.png}
    \end{subfigure}
    \begin{subfigure}{4cm}
      \includegraphics[width=4cm,height=4cm]{images/cartesian_coordinate_system_3d.png}
    \end{subfigure}
  \end{figure}

\end{frame}

\begin{frame}{Matrice}

  \begin{itemize}
    \item Matrica je tabela skalara
    \item Pogodna za opisivanje matematičkih objekata i njihovih osobina
    \item Matrica ima $N\times M$ skalara, gde je $N$ broj redova, a $M$ broj kolona
  \end{itemize}
  \begin{figure}
    \begin{subfigure}{3cm}
      $\begin{bmatrix}
        1 & 2 & 3\\
        4 & 5 & 6
        \end{bmatrix}$
    \end{subfigure}
    \begin{subfigure}{3cm}
      $\begin{bmatrix}
        -1 & 3 & 1\\
        4 & 5 & 10
        \end{bmatrix}$
    \end{subfigure}
    \begin{subfigure}{3cm}
      $\begin{bmatrix}
        0 & 1.2 & 1\\
        1 & -2 & 1.1
        \end{bmatrix}$
    \end{subfigure}
  \end{figure}

\end{frame}

\begin{frame}{Matrice - Osnovne operacije i pojmovi}

  Sabiranje:
  \begin{figure}
    \begin{subfigure}{9cm}
      $\begin{bmatrix}
        1 & 2 & 3\\
        4 & 5 & 6
        \end{bmatrix}$
      + 
      $\begin{bmatrix}
        10 & -3 & 1\\
        -3 & 2 & 1
        \end{bmatrix}$
      =
      $\begin{bmatrix}
        11 & -1 & 4\\
        1 & 7 & 7
        \end{bmatrix}$
    \end{subfigure}
  \end{figure}

  Oduzimanje:
  \begin{figure}
    \begin{subfigure}{9cm}
      $\begin{bmatrix}
        1 & 2 & 3\\
        4 & 5 & 6
        \end{bmatrix}$
      - 
      $\begin{bmatrix}
        10 & -3 & 1\\
        -3 & 2 & 1
        \end{bmatrix}$
      =
      $\begin{bmatrix}
        -9 & 5 & 2\\
        7 & 3 & 5
        \end{bmatrix}$
    \end{subfigure}
  \end{figure}

  Množenje skalarom:
  \begin{figure}
    \begin{subfigure}{9cm}
      
      $
      3\times
      \begin{bmatrix}
        10 & -3 & 1\\
        -3 & 2 & 1
        \end{bmatrix}$
      =
      $\begin{bmatrix}
        30 & -9 & 3\\
        -9 & 6 & 3
        \end{bmatrix}$
    \end{subfigure}
  \end{figure}

\end{frame}

\begin{frame}{Matrice - Osnovne operacije i pojmovi (2)}

  Množenje po elementima:
  \begin{figure}
    \begin{subfigure}{9cm}
      $\begin{bmatrix}
        1 & 2 & 3\\
        4 & 5 & 6
        \end{bmatrix}$
      * 
      $\begin{bmatrix}
        10 & -3 & 1\\
        -3 & 2 & 1
        \end{bmatrix}$
      =
      $\begin{bmatrix}
        10 & -6 & 3\\
        -12 & 10 & 6
        \end{bmatrix}$
    \end{subfigure}
  \end{figure}

  Nula matrica (neutral za sabiranje i oduzimanje):
  \begin{figure}
    \begin{subfigure}{9cm}
      $\begin{bmatrix}
        0 & 0 & 0\\
        0 & 0 & 0
        \end{bmatrix}$
    \end{subfigure}
  \end{figure}

\end{frame}

\begin{frame}{Matrice - Osnovne operacije i pojmovi (3)}

  Množenje:
  \begin{figure}
    \begin{subfigure}{9cm}
      $\begin{bmatrix}
        1 & 2 & 3\\
        4 & 5 & 6
        \end{bmatrix}$
      * 
      $\begin{bmatrix}
        1 & 2\\
        3 & 4\\
        5 & 6
        \end{bmatrix}$
      =
      $\begin{bmatrix}
        22 & 28\\
        49 & 64
        \end{bmatrix}$
    \end{subfigure}
  \end{figure}

  Postupno:
  \begin{figure}
    \begin{subfigure}{9cm}
      $\begin{bmatrix}
        1*1 + 2*3 + 3*5 & 1*2 + 2*4 + 3*6\\
        4*1 + 5*3 + 6*5 & 4*2 + 5*4 + 6*6
      \end{bmatrix}$
    \end{subfigure}
  \end{figure}
  \begin{itemize}
    \item Polje \textit{[i, j]} u rezultujućoj matrici se dobija kao skalarni proizvod \textit{i}-tog reda i \textit{j}-te kolone
    \item Množenjem matrice dimenzije $N \times M$ i $M \times K$ se dobija matrica dimenzije $N \times K$
    \item Množenje je definisano samo ako se broj kolona leve matrice i broj redova desne matrice poklapa!
  \end{itemize}

\end{frame}

\begin{frame}{Matrice - Osnovne operacije i pojmovi (4)}

  Množenje matrica nije komutativno (zamenom mesta operanada se ne dobija uvek isti rezultat):
  \begin{figure}
    \begin{subfigure}{9cm}
      $\begin{bmatrix}
        1 & 2\\
        3 & 4\\
        5 & 6
        \end{bmatrix}$
      * 
      $\begin{bmatrix}
        1 & 2 & 3\\
        4 & 5 & 6
        \end{bmatrix}$
      =
      $\begin{bmatrix}
        9 & 12 & 15\\
        19 & 26 & 33\\
        29 & 40 & 51
        \end{bmatrix}$
    \end{subfigure}
  \end{figure}
  \begin{itemize}
    \item Zamenom redosleda operadana u opštem slučaju može da se dobija nedefinisana operacija
    \item Primer: Množenje matrica dimenzija (10, 2) i (2, 3) je definisano, ali (2, 3) i (10, 2) nije
  \end{itemize}

\end{frame}

\begin{frame}{Matrice - Transponovane matrice}

  Transponovana matrica $A^{T}$ matrice $A$ se dobija zamenom kolona i redova:
  \begin{figure}
    \begin{subfigure}{9cm}
      $\begin{bmatrix}
        1 & 2\\
        3 & 4\\
        5 & 6
        \end{bmatrix}^{T}$
      =
      $\begin{bmatrix}
        1 & 2 & 3\\
        4 & 5 & 6
        \end{bmatrix}$
    \end{subfigure}
  \end{figure}

  \begin{figure}
    \begin{subfigure}{9cm}
      $\begin{bmatrix}
        1 & 2
        \end{bmatrix}^{T}$
      =
      $\begin{bmatrix}
        1\\
        2
        \end{bmatrix}$
    \end{subfigure}
  \end{figure}

\end{frame}

\begin{frame}{Kvadratne matrice}

  Matrica je kvadratna ako ima jednak broj redova i kolona
  \begin{figure}
    \begin{subfigure}{9cm}
      $\begin{bmatrix}
        1 & 2\\
        3 & 4\\
        \end{bmatrix}$,
        $\begin{bmatrix}
          1 & 2 & 10\\
          3 & 4 & 30\\
          5 & 6 & 100
          \end{bmatrix}$,
        $\begin{bmatrix}
          100
          \end{bmatrix}$
    \end{subfigure}
  \end{figure}
  \begin{itemize}
    \item Množenjem dve kvadratne matrice iste dimenzije dobijamo kvadratnu matricu iste dimenzije
    \item Ako permutujemo operadne, množenje je i dalje definisano (ali operacija i dalje nije komutativna)
  \end{itemize}

\end{frame}

\begin{frame}{Kvadratne matrice - Neutral i inverz za množenje}
  Za kvadratne matrice postoji neutral za množenje:
  \begin{figure}
    \begin{subfigure}{7cm}
      $I_{2}$ = 
      $\begin{bmatrix}
        1 & 0\\
        0 & 1\\
        \end{bmatrix}$,
      \hfill
      $A \times I = I \times A = A$
    \end{subfigure}
  \end{figure}

  \begin{figure}
    \begin{subfigure}{10cm}
      $\begin{bmatrix}
        1 & 2\\
        3 & 4\\
        \end{bmatrix}$
      *
      $\begin{bmatrix}
        1 & 0\\
        0 & 1\\
        \end{bmatrix}$
      = 
      $\begin{bmatrix}
        1*1 + 2*0 & 1*0 + 2*1\\
        3*1 + 4*0 & 3*0 + 4*1\\
        \end{bmatrix}$
      = 
      $\begin{bmatrix}
        1 & 2\\
        3 & 4\\
        \end{bmatrix}$
    \end{subfigure}
  \end{figure}
  Takođe može postoji i inverz (ali i ne mora) za kvadratnu matricu:
  \begin{figure}
    \begin{subfigure}{5cm}
      $A \times A^{-1} = A^{-1} \times A = I$
    \end{subfigure}
  \end{figure}
  Matrice koje nemaju inverz su \textbf{singularne}

\end{frame}

\begin{frame}{Vektori kao specijalni slučajevi matrica}

  Vektore možemo da posmatramo kao specijalan slučaj matrica dimenzije $N \times 1$
  \begin{figure}
    \begin{subfigure}{10cm}
      $\begin{bmatrix}
        1\\
        3\\
        \end{bmatrix}$
      $\begin{bmatrix}
        -1\\
        3\\
        \end{bmatrix}$
      $\begin{bmatrix}
        1.2\\
        3.3\\
        \end{bmatrix}$
    \end{subfigure}
  \end{figure}
  Možemo i da množimo vektore matricama:
  \begin{figure}
    \begin{subfigure}{5cm}
      $\begin{bmatrix}
        1 & 1\\
        -1 & 1\\
        \end{bmatrix}$
      *
      $\begin{bmatrix}
        1\\
        3\\
        \end{bmatrix}$
      =
      $\begin{bmatrix}
        4\\
        2\\
        \end{bmatrix}$
    \end{subfigure}
  \end{figure}
  U slučaju da vektori predstavljaju tačke, onda množenjem tih vektora matricama vršimo transformaciju tačaka u prostoru
  
\end{frame}

\section{Afine transformacije u ravni}

\begin{frame}{Koordinatni sistem slika}

  Koordinatni sistem slike se posmatra od gornjeg levog ugla, gde je $y$ osa orijentisana ka dole:
  \begin{figure}
    \begin{subfigure}{5cm}
      \includegraphics[width=5cm,height=5cm]{images/01_01_example_before.png}
    \end{subfigure}
    \begin{subfigure}{5cm}
      \begin{itemize}
        \item Koordinata $(0, 0)$ u gornjem levom uglu odgovara zelenoj tački (centru koordinatnog sistema slike)
        \item Tačka u donjem desnom uglu ima koordinatu $(H-1, W-1)$, gde je $H\times W$ rezolucija slike
      \end{itemize}
    \end{subfigure}
  \end{figure}
  
\end{frame}

\begin{frame}{Linearne transformacije}

  Linearna funkcija u jednoj dimenzija je funkcija oblika: 
  \begin{itemize}
    \item $f(x) = a\cdot x + b$
    \item Primer: 
      \begin{itemize}
        \item $f(x) = 2\cdot x + 3$, 
        \item $f(4) = 2\cdot 4 + 3 = 11$
      \end{itemize}
  \end{itemize}
  U ravni (2D prostor) možemo da izvršimo transformaciju korišćenjem dve funkcije $f$ i $g$:
  \begin{itemize}
    \item $\hat{x} = f(x, y) = a\cdot x + b\cdot y + t_{x}$
    \item $\hat{y} = g(x, y) = c\cdot x + d\cdot y + t_{y}$
  \end{itemize}
  Gde su $x$ i $y$ koordinate pre transformacije $(f, g)$, a $\hat{x}$ i $\hat{y}$ koordinate nakon transformacije.

\end{frame}

\begin{frame}{Linearne transformacije (2)}

  \begin{itemize}
    \item $\hat{x} = f(x, y) = a\cdot x + b\cdot y + t_{x}$
    \item $\hat{y} = g(x, y) = c\cdot x + d\cdot y + t_{y}$
  \end{itemize}
  Kombinaciju $(f, g)$ možemo kompaktnije da zapišemo pomoću vektora i matrica:
  \begin{figure}
    \begin{subfigure}{9cm}
      $\begin{bmatrix}
        \hat{x}\\
        \hat{y}
        \end{bmatrix}$
      =
      $\begin{bmatrix}
        a & b\\
        c & d
        \end{bmatrix}$
      *
      $\begin{bmatrix}
        x\\
        y
        \end{bmatrix}$
      + 
      $\begin{bmatrix}
        t_{x}\\
        t_{y}
        \end{bmatrix}$
      = 
      $\begin{bmatrix}
        a\cdot x + b\cdot y + t_{1}\\
        c\cdot x + d\cdot y + t_{2}
        \end{bmatrix}$
    \end{subfigure}
  \end{figure}
  Prethodnu formulu nazivamo \textbf{transformacija koordinata tačaka u ravni}, a vektor $(t_{x}, t_{y})$ nazivamo vektorom translacije

\end{frame}

\begin{frame}{Određivanje transformacije na osnovu uparenih tačaka}

  \begin{figure}
    \begin{subfigure}{5cm}
      $\begin{bmatrix}
        \hat{x}\\
        \hat{y}
        \end{bmatrix}$
      =
      $\begin{bmatrix}
        a & b\\
        c & d
        \end{bmatrix}$
      *
      $\begin{bmatrix}
        x\\
        y
        \end{bmatrix}$
      + 
      $\begin{bmatrix}
        t_{x}\\
        t_{y}
        \end{bmatrix}$
    \end{subfigure}
  \end{figure}
  Koliko uparenih tačaka $(x, y)$ i $(\hat{x}, \hat{y})$ nam je neophodno da odredimo komponente transformacije?
  \begin{itemize}
    \item Imamo 6 parametara (6 nepoznatih)
    \item Svaki par uparenih tačaka $(x, y)$ i $(\hat{x}, \hat{y})$ nam daje dve jednačine (jednu po $x$ i jednu po $y$ koordinati)
    \item To znači da nam je potrebno da imamo 3 para uparenih tačaka
  \end{itemize}

\end{frame}

\begin{frame}{Translacija}

  Translacija predstavlja pomeranje tačke za vektor T = $(t_{x}, t_{y})$
  \begin{figure}
    \begin{subfigure}{8cm}
      $\begin{bmatrix}
        \hat{x}\\
        \hat{y}
        \end{bmatrix}$
      =
      $\begin{bmatrix}
        1 & 0\\
        0 & 1
        \end{bmatrix}$
      *
      $\begin{bmatrix}
        x\\
        y
        \end{bmatrix}$
      + 
      $\begin{bmatrix}
        t_{x}\\
        t_{y}
        \end{bmatrix}$
      =
      $\begin{bmatrix}
        x + t_{x}\\
        y + t_{y}
        \end{bmatrix}$
    \end{subfigure}
  \end{figure}

  Primer: Translacija slike za vektor $(100, 200)$:
  \begin{figure}
    \begin{subfigure}{5cm}
      \includegraphics[width=5cm,height=5cm]{images/01_01_example_before.png}
    \end{subfigure}
    \begin{subfigure}{5cm}
      \includegraphics[width=5cm,height=5cm]{images/01_02_example_after.png}
    \end{subfigure}
  \end{figure}

\end{frame}

\begin{frame}{Rotacija}

  Rotacija je malo komplikovanija, ali i dalje možemo da izvedemo formulu pomoću trigonometrije
  \begin{figure}
    \begin{subfigure}{3cm}
      \includegraphics[width=3cm,height=3cm]{images/rotate_points.png}
    \end{subfigure}
    \hfill
    \begin{subfigure}{6cm}
      $\begin{bmatrix}
        \hat{x}\\
        \hat{y}
        \end{bmatrix}$
      =
      $\begin{bmatrix}
        cos\theta & -sin\theta\\
        sin\theta & cos\theta
        \end{bmatrix}$
      *
      $\begin{bmatrix}
        x\\
        y
        \end{bmatrix}$
    \end{subfigure}
  \end{figure}
  
  \begin{figure}
    \begin{subfigure}{5cm}
      \includegraphics[width=3cm,height=3cm]{images/02_01_example_before.png}
    \end{subfigure}
    \begin{subfigure}{5cm}
      \includegraphics[width=3cm,height=3cm]{images/02_02_example_after.png}
    \end{subfigure}
  \end{figure}

\end{frame}

\begin{frame}{Rotacija oko tačke}

  \begin{itemize}
    \item Prethodni primer je rotacija oko koordinatnog početka koji se nalazi u gornjem levom uglu
    \item Ako želimo da rotiramo oko centra slike, moramo da pomerimo centar u koordinatni sistem,
          izvršimo transformaciju i onda vratimo centar na prvobitnu poziciju
  \end{itemize}
  
  \begin{figure}
    \begin{subfigure}{5cm}
      \includegraphics[width=5cm,height=5cm]{images/02_01_example_before.png}
    \end{subfigure}
    \begin{subfigure}{5cm}
      \includegraphics[width=5cm,height=5cm]{images/02_03_example_after.png}
    \end{subfigure}
  \end{figure}

\end{frame}

\begin{frame}{Skaliranje}

  \begin{itemize}
    \item Skaliranje menja dimenzije objekata na sceni.
    \item Ukoliko je podjednako skaliranje po $x$ i po $y$ osi, onda se ta transformacija zove \textit{homotetija}.
  \end{itemize}
  \begin{figure}
    \begin{subfigure}{9cm}
      $\begin{bmatrix}
        \hat{x}\\
        \hat{y}
        \end{bmatrix}$
      =
      $\begin{bmatrix}
        \lambda_{x} & 0\\
        0 & \lambda_{y}
        \end{bmatrix}$
      *
      $\begin{bmatrix}
        x\\
        y
        \end{bmatrix}$,
      \hfill
      $S_{\lambda_{x}, \lambda_{y}} =$
      $\begin{bmatrix}
        \lambda_{x} & 0\\
        0 & \lambda_{y}
        \end{bmatrix}$
    \end{subfigure}
  \end{figure}
  
  \begin{figure}
    \begin{subfigure}{5cm}
      \includegraphics[width=5cm,height=5cm]{images/03_01_example_before.png}
    \end{subfigure}
    \begin{subfigure}{5cm}
      \includegraphics[width=5cm,height=5cm]{images/03_02_example_after.png}
    \end{subfigure}
  \end{figure}

\end{frame}

\begin{frame}{Skaliranje oko tačke}

  \begin{itemize}
    \item Isto kao i rotacija oko tačke, možemo da izvršimo skaliranje u odnosu na center slike umesto gornji levi ćošak
  \end{itemize}
  
  \begin{figure}
    \begin{subfigure}{5cm}
      \includegraphics[width=5cm,height=5cm]{images/03_01_example_before.png}
    \end{subfigure}
    \begin{subfigure}{5cm}
      \includegraphics[width=5cm,height=5cm]{images/03_03_example_after.png}
    \end{subfigure}
  \end{figure}

\end{frame}

\begin{frame}{Refleksija}

  Varijante:
  \begin{itemize}
    \item Refleksija oko $x$ ose
    \item Refleksija oko $y$ ose
    \item Refleksija oko koordinatnog početka (kompozicija prethodna dva)
    \item Refleksija oko tačke (centra slike)
  \end{itemize}
  \begin{figure}
    \begin{subfigure}{9cm}
      $R_{x} =$
      $\begin{bmatrix}
        -1 & 0\\
        0 & 1
      \end{bmatrix}$,
      \hfill
      $R_{y} =$
      $\begin{bmatrix}
        1 & 0\\
        0 & -1
      \end{bmatrix}$,
      \hfill
      $R_{o} =$
      $\begin{bmatrix}
          -1 & 0\\
          0 & -1
      \end{bmatrix}$
    \end{subfigure}
  \end{figure}
  
  \begin{figure}
    \begin{subfigure}{3cm}
      \includegraphics[width=3cm,height=3cm]{images/04_02_example_after.png}
    \end{subfigure}
    \begin{subfigure}{3cm}
      \includegraphics[width=3cm,height=3cm]{images/04_03_example_after.png}
    \end{subfigure}
    \begin{subfigure}{3cm}
      \includegraphics[width=3cm,height=3cm]{images/04_04_example_after.png}
    \end{subfigure}
  \end{figure}

\end{frame}

\section{Model kamere kroz rupicu\\(\textit{Pinhole Camera Model})}

\begin{frame}{Projekcija}

  \begin{itemize}
    \item Projekcija je preslikavanje jednog prostora dimenzije $N$ u prostor dimenzije manje od $N$
    \item Primer: Preslikavanje objekta iz stvarnog sveta (3D) na sliku (2D)
  \end{itemize}
  \begin{figure}
    \begin{subfigure}{6cm}
      \includegraphics[width=6cm,height=4cm]{images/projection.png}
    \end{subfigure}
  \end{figure}

\end{frame}

\begin{frame}{Model kamere kroz rupicu}

  \begin{figure}
    \begin{subfigure}{10cm}
      \includegraphics[width=10cm,height=5cm]{images/pinhole_camera_1.png}
    \end{subfigure}
  \end{figure}

\end{frame}

\begin{frame}{Model kamere kroz rupicu - Formalna reprezentacija}

  \begin{figure}
    \begin{subfigure}{7cm}
      \includegraphics[width=7cm,height=3cm]{images/pinhole_camera_2.png}
    \end{subfigure}
  \end{figure}
  \begin{itemize}
    \item Kamera se nalazi u koordinatnom početku \textit{O} referentnog koordinatnog sistema \textit{(i, j, k)}
    \item Centar slike \textit{C'} se nalazi na osi \textit{k} na udaljenosti \textit{f} od tačke \textit{O}
    \item Tačka $P (x, z, y)$ je tačka u realnom svetu koja se projektuje u tačku $P^{\prime} (\hat{x}, \hat{y})$ na slici 
  \end{itemize}
\end{frame}

\begin{frame}{Model kamere kroz rupicu - Veza između \textit{P} i \textit{P'}}

  \begin{itemize}
    \item Neka je tačka $P^{\prime \prime}$ dobijena projekcijom tačke $P$ na osu $k$
    \item Trouglovi $\triangle P^{\prime}P^{\prime \prime} O$ i $\triangle P^{\prime} C^{\prime} O$ su slični
    \item Dobijamo sledeću jednakost: $P^{\prime \prime}O : OC^{\prime} = PP^{\prime \prime} : P^{\prime} C^{\prime}$
    \item Sledeće važi: $P^{\prime \prime}O = z$, $OC^{\prime} = f$, $PP^{\prime \prime} = y$, $P^{\prime} C^{\prime} = \hat{y}$
    \item Odatle sledi: $z : f = y : \hat{y}$ tj. $\hat{y} = \frac{y}{z} \cdot f$
    \item Analogno važi i: $\hat{x} = \frac{x}{z} \cdot f$
  \end{itemize}
  Ako su nam poznate koordinate tačke P, \textit{C'} i vrednost \textit{f}, onda možemo da odredimo i tačku \textit{P'}

\end{frame}

\begin{frame}{Model kamere kroz rupicu - Udaljenost od kamere}

  \begin{itemize}
    \item $\hat{y} = \frac{y}{z} \cdot f$, $\hat{x} = \frac{x}{z} \cdot f$
    \item Što je \textit{z} veće (što je objekat u stvarnom svetu dalji) to su koordinate $\hat{x}$ i $\hat{y}$ manje (to je objekat na slici manji)
  \end{itemize}
  \begin{figure}
    \begin{subfigure}{7cm}
      \includegraphics[width=7cm,height=5cm]{images/pisa.jpg}
    \end{subfigure}
  \end{figure}

\end{frame}

\begin{frame}{Model kamere kroz rupicu - Kompaktan zapis preslikavanja}

  Da li možemo da preslikavanje \textit{P} u \textit{P'} napišemo kompaktno preko množenja matrice?
  \begin{figure}
    \begin{subfigure}{6cm}
      $\begin{bmatrix}
        \hat{x}\\
        \hat{y}
      \end{bmatrix}$
      = $\begin{bmatrix}
        \frac{x}{z} \cdot f\\
        \frac{y}{z} \cdot f
      \end{bmatrix}$
      = M $\times$
      $\begin{bmatrix}
        x\\
        y\\
        z
      \end{bmatrix}$
    \end{subfigure}
  \end{figure}
  U ovakvom obliku ne možemo. Možda ako promenimo format nekih tačaka?


\end{frame}

\begin{frame}{Homogene koordinate u 2D}

\begin{itemize}
  \item \textbf{Homogena tačka} koja odgovara tački $(x, y)$ u ravni je $(x, y, 1) \tilde{=} \alpha \cdot (x, y, 1)$ 
  \item Homogene tačke imaju skup osobina koje ih čine pogodnije za rad u odnosu na tačke Kartezijevog koordinatnog sistema
  \item U projektivnoj geometriji se uglavnom radi sa homogenim koordinatama, jer na elegantan način predstavljamo
        vezu između stvarnog sveta i projektivne ravni
\end{itemize}

\end{frame}

\begin{frame}{Homogene koordinate u 2D - Osobine}

  \begin{itemize}
    \item Za fiksirano $\alpha$ geometrijski smisao izraza $\alpha (x, y, 1)$ je skup tačaka na pravi koje se slikaju u istu tačku na 
          projektivnoj ravni (slici)
    \item Homogene koordinate možemo jednostavno preslikavamo u ,,obične`` koordinate:
      \begin{itemize}
        \item Primer: $A_{h} = (a, b, c)$
        \item $x = \frac{a}{c}$
        \item $y = \frac{b}{c}$
        \item Šta ako je $c = 0$? Tada kažemo da se ta tačka nalazi u beskonačnosti.
      \end{itemize}
    \item U Projektivnog geometriji se sve paralelne prave seku u beskonačnosti
  \end{itemize}
  \begin{figure}
    \begin{subfigure}{4cm}
      \includegraphics[width=4cm,height=4cm]{images/perspective_railway.png}
    \end{subfigure}
  \end{figure}
  
\end{frame}

\begin{frame}{Model kamere kroz rupicu - Osnovni oblik}

  Da li možemo da preslikavanje \textit{P} u \textit{P'} napišemo kompaktno preko množenja matrice?
  \begin{figure}
    \begin{subfigure}{6cm}
      $\begin{bmatrix}
        \hat{x}\\
        \hat{y}\\
        1
      \end{bmatrix}$
      = $\alpha \cdot
      \begin{bmatrix}
        x \cdot f\\
        y \cdot f\\
        z
      \end{bmatrix}$
      = M $\times$
      $\begin{bmatrix}
        x\\
        y\\
        z
      \end{bmatrix}$
    \end{subfigure}
  \end{figure}
  \begin{itemize}
    \item Vrednost za $\alpha$ je $\frac{1}{z}$ kada uporedimo sa prethodnim jednačinama
    \item Lako se proverava da M (matrica kamere) ima sledeći oblik
  \end{itemize}

  \begin{figure}
    \begin{subfigure}{4cm}
      M =
      $\begin{bmatrix}
        f & 0 & 0\\
        0 & f & 0\\
        0 & 0 & 1
      \end{bmatrix}$
    \end{subfigure}
  \end{figure}

\end{frame}

\begin{frame}{Model kamere kroz rupicu - Pikseli}

  \begin{itemize}
    \item Prethodni oblik matrice kamere ne uzima u obzir da je slika predstavljana pikselima (kvadratićima), 
    a stvarni svet je u centimetrima
    \item Neophodno je da uvedemo preslikavanje kao razmera piksela i centimetara
    \item Ta razmera nije nužno ista po $x$ i po $y$ osi (pikseli ne moraju da budu kvadratići)
    \item Definišemo te vrednosti kao $r_{x}$ i $r_{y}$
  \end{itemize}

  \begin{figure}
      \begin{subfigure}{4cm}
        M =
        $\begin{bmatrix}
          f\cdot r_{x} & 0 & 0\\
          0 & f\cdot r_{y} & 0\\
          0 & 0 & 1
        \end{bmatrix}$
      \end{subfigure}
    \end{figure}
\end{frame}

\begin{frame}{Model kamere kroz rupicu - Translacija koordinatnog početka slike}

  \begin{itemize}
    \item Do sada smo pretpostavljali da se koordinatni početak slike nalazi u samom centru slike
    \item Kao što smo videli, na računarima se uglavnom tako ne predstavljaju slike, već se uzima levi gornji (donji) ćošak
    \item U odnosu na taj ćošak, sve tačke koje se projektuju na sliku su translirane za vektor $(c_{x}, c_{y})$ do centra \textit{C'}
    \item Prednost homogenih koordinata je što translaciju možemo kompaktnije da zapišemo u okviru $3 \times 3$ matrice
  \end{itemize}

  \begin{figure}
      \begin{subfigure}{4cm}
        M =
        $\begin{bmatrix}
          f\cdot r_{x} & 0 & c_{x}\\
          0 & f\cdot r_{y} & c_{y}\\
          0 & 0 & 1
        \end{bmatrix}$
      \end{subfigure}
    \end{figure}
\end{frame}

\begin{frame}{Model kamere kroz rupicu - Unutrašnji parametri}

  \begin{figure}
      \begin{subfigure}{4cm}
        M =
        $\begin{bmatrix}
          f\cdot r_{x} & s & c_{x}\\
          0 & f\cdot r_{y} & c_{y}\\
          0 & 0 & 1
        \end{bmatrix}$
      \end{subfigure}
    \end{figure}

  \begin{itemize}
    \item Parameter $s$ je parameter smicanja (uglavnom teži nula kod kamera)
    \item Prethodna matrica predstavlja pojednostavljeni oblik unutrašnje parametre kamere (ne uzimamo u obzir distorziju)
    \item Unutrašnji parametri su dovoljni u situacijama kada se koordinatni sistem kamere i stvarnog sveta poklapaju
    \item To je veoma jaka pretpostavka i skoro nikada ne važi
  \end{itemize}
\end{frame}

\begin{frame}{Model kamere kroz rupicu - Spoljašnji parametri}

  Preslikavanje koordinatnog sistema stvarnog sveta u koordinatni sistem kamera možemo da izvedemo kompozicijom rotacije 
  \textit{R} i translacije \textit{T}
  \begin{itemize}
    \item Transliramo koordinatni početak kamere u koordinatni početak stvarnog sveta
    \item Modifikujemo orijentaciju rotacijom
  \end{itemize}

  \begin{figure}
    \begin{subfigure}{7cm}
      \includegraphics[width=7cm,height=5cm]{images/intrinsic_parameters.png}
    \end{subfigure}
  \end{figure}

\end{frame}

\begin{frame}{Kompozicija transformacija i množenje matrica}

  Kompoziciju transformacija možemo da izvedemo množenjem odgovarajućih
  matrica transformacija (kompozicija rotacije i translacije):
  \begin{figure}
    \begin{subfigure}{10cm}
      $\begin{bmatrix}
        1 & 0 & 3\\
        0 & 1 & 4\\
        0 & 0 & 1
      \end{bmatrix}$
      $\times$
      $\begin{bmatrix}
        cos\frac{\pi}{3} & -sin\frac{\pi}{3} & 0\\
        sin\frac{\pi}{3} & cos\frac{\pi}{3} & 0\\
        0 & 0 & 1
      \end{bmatrix}$
      =
      $\begin{bmatrix}
        cos\frac{\pi}{3} & -sin\frac{\pi}{3} & 3\\
        sin\frac{\pi}{3} & cos\frac{\pi}{3} & 4\\
        0 & 0 & 1
      \end{bmatrix}$
    \end{subfigure}
  \end{figure}

\end{frame}

\begin{frame}{Model kamere kroz rupicu - Spoljašnji parametri (2)}

  \begin{itemize}
    \item Pretpostavljamo da su nam poznate koordinate $P_{cam}$ kamere u stvarnom koordinatnom sistemu i da nam je poznata
          rotacija koja dovodi orijentaciju koordinatnog sistema stvarnog sveta u koordinatni sistem kamere
    \item Primenom translacije -$P_{cam}$ dovodimo koordinatni početak kamere u koordinatni početak stvarnog sveta
    \item Primenom rotacije nakon toga usklađujemo orijentacije
    \item Primenom matrice M nakon toga vršimo projekciju na sliku
  \end{itemize}

  \begin{figure}
    \begin{subfigure}{3cm}
      $K = R \times T_{-P_{cam}}$
    \end{subfigure}
  \end{figure}

  \begin{itemize}
    \item Napomena: Ove transformacije su u $3 \times 3$ dimenzije
    \item Da bi se translacija u 3D svetu svela na množenje matrica moraju da se koriste homogene koordinate za prostor (vektori dužine 4)
  \end{itemize}

\end{frame}

\begin{frame}{Model kamere kroz rupicu - Konačan oblik matrice kamere}

  Konačan oblik jednačine kamere:
  \begin{figure}
    \begin{subfigure}{6cm}
      $\begin{bmatrix}
        \hat{x}\\
        \hat{y}\\
        1
      \end{bmatrix}$
      = $M \times R \times T_{-P_{cam}}$
      $\times$
      $\begin{bmatrix}
        x\\
        y\\
        z\\
        1
      \end{bmatrix}$
    \end{subfigure}
  \end{figure}
  \begin{itemize}
    \item $M$ je matrica unutrašnjih parametara
    \item $R$ i $T_{-P_{cam}}$ su spoljašnji parametri
    \item Matrica $M \times R \times T_{-P_{cam}}$ je matrica dimenzije $3 \times 4$
  \end{itemize}

\end{frame}

\begin{frame}{Kalibracija kamere}

  \textbf{Kalibracija kamare} je proces dobijanja unutrašnjih i spoljašnjih parametara kamere tj. dobijanja matrice kamere \textit{K}
  \begin{itemize}
    \item Matrica \textit{K} ima $3 \times 4 = 12$ parametara (11 ako $p_{34}$ fiksiramo na 1)
    \item Dekomponovana matrica se sastoji iz $M$ (5 parametra), $R$ (3 parametra) i $T_{-P_{cam}}$ (3 parametra)
  \end{itemize}
  \begin{figure}
    \begin{subfigure}{7cm}
      $\begin{bmatrix}
        \hat{x}\\
        \hat{y}\\
        1
      \end{bmatrix}$
      =
      $\begin{bmatrix}
        p_{11} & p_{12} & p_{13} & p_{14}\\
        p_{21} & p_{22} & p_{23} & p_{24}\\
        p_{31} & p_{32} & p_{33} & p_{34}\\
      \end{bmatrix}$
      $\times$
      $\begin{bmatrix}
        x\\
        y\\
        z\\
        1
      \end{bmatrix}$
    \end{subfigure}
  \end{figure}
  \begin{itemize}
    \item Za svaki par uparenih tačaka iz stvarnog sveta i slike imamo 2 jednačine
    \item Ako imamo 11 nepoznatih paramatera, onda nam je potrebno barem 11 jednačina (6 parova) - DLT algoritam
  \end{itemize}

\end{frame}

\begin{frame}{Kalibracija kamere - Zengova metoda}

  \textbf{Zengova metoda} nam omogućava da izvršimo kalibraciju kamere korišćenjem slike šahovske table

  \begin{itemize}
    \item Kao osnovu koristi DLT algoritam
    \item Glavna pretpostavka je da su sve tačke na slici u istoj ravni
    \item Neophodno je barem 3 slika istih rezolucija iz različitih uglova (što više to bolje)
  \end{itemize}
  \begin{figure}
    \begin{subfigure}{7cm}
      \includegraphics[width=7cm,height=5cm]{images/zhang_method.png}
    \end{subfigure}
  \end{figure}

\end{frame}

\begin{frame}{Kalibracija kamere - Zengova metoda (2)}

  Zašto šahovska tabla? 
  \begin{itemize}
    \item Jednostavno je da se detektuju unutrašnji čoškovi zbog velike kontrasti
    \item Kao dodatna olakšica za detekciju je unos dimenzije same šahovske (u odnosu na unutrašnje uglove)
  \end{itemize}

  \begin{figure}
    \begin{subfigure}{7cm}
      \includegraphics[width=7cm,height=5cm]{images/findchessboardcorners.jpeg}
    \end{subfigure}
  \end{figure}

\end{frame}

\begin{frame}{Literatura}

  \begin{itemize}
    \item \href{https://web.stanford.edu/class/cs231a/}{Stanford CS231A: Computer Vision, From 3D Reconstruction to Recognition (Winter 2022)}
    \item Programming Computer Vision with Python, Jan Erik Solem
    \item Multiple View Geometry in Computer Vision, Richard Hartley, Andrew Zisserman
    \item Geometrija za informatičare, Tijana Šukilović, Srđan Vukmirović
  \end{itemize}

\end{frame}


\end{document}